
%%
%% This is file `./samples/longsample.tex',
%% generated with the docstrip utility.
%%
%% The original source files were:
%%
%% apa6.dtx  (with options: `longsample')
%% ----------------------------------------------------------------------
%% 
%% apa6 - A LaTeX class for formatting documents in compliance with the
%% American Psychological Association's Publication Manual, 6th edition
%% 
%% Copyright (C) 2011-2016 by Brian D. Beitzel <brian at beitzel.com>
%% 
%% This work may be distributed and/or modified under the
%% conditions of the LaTeX Project Public License (LPPL), either
%% version 1.3c of this license or (at your option) any later
%% version.  The latest version of this license is in the file:
%% 
%% http://www.latex-project.org/lppl.txt
%% 
%% Users may freely modify these files without permission, as long as the
%% copyright line and this statement are maintained intact.
%% 
%% This work is not endorsed by, affiliated with, or probably even known
%% by, the American Psychological Association.
%% 
%% ----------------------------------------------------------------------
%% 
\documentclass[man]{apa6}

\usepackage[american]{babel}
\usepackage[utf8]{inputenc}
\usepackage[T1]{fontenc}
\usepackage{float}
\usepackage{placeins}

\usepackage{csquotes}
\usepackage[style=apa,sortcites=true,sorting=nyt,backend=biber]{biblatex}
\DeclareLanguageMapping{american}{american-apa}
\addbibresource{bibliography.bib}

\title{Plan of Approach: Setting up HR Policies in a startup in Curitiba, Brazil}
\shorttitle{Setting up HR Policies}

\author{Cas de Groot (496600)}
\affiliation{HAN University of Applied Sciences}

%\leftheader{Beitzel}

%\abstract{}

%\keywords{APA style, demonstration}

\authornote{}

\begin{document}
\maketitle
\tableofcontents
\clearpage
\section{Situational Description}
\subsection{The company}
The client company, Instok, is a startup company in Curitiba, Brazil. Currently it consists of 4 members with the following roles: CEO, CFO, Marketing, and Sales. It was founded in November 2016 and is currently preparing its product for launch (planned July 2017). The company has been constructed and is being managed based on the methods explained in the book "The Lean Startup: How Today's Entrepreneurs Use Continuous Innovation to Create Radically Successful Businesses" by Eric Ries. 

\subsubsection{The product}
Instok will provide their customers with a mobile application and website that allows customers to browse through a product catalogue that features products from store warehouses (so the products that are on display inside the stores, but are still in stock). It does this from the perspective of reducing sleeping capital for stores, while allowing the application to function as an online outlet store. This allows the customer to buy out of season clothes that are left over from the collection for a lower price, and the store owners are able to sell their products in stock which would otherwise have been sitting in the warehouse without any exposure to customers.\\
According to research of Instok, this is an entirely untapped market. Conversations with store owners and managers have shown that this is a problem that has existed for many years, and nobody has found a sustainable solution. Some of the larger chains have set up their own outlets, but many smaller store chains and privately owned stores don't have the capital to do this. For them items in their stock are sleeping capital \parencite{SNELL2009} and some store managers have mentioned that there are items in their stock that have been there for over a year.\\
Instok's product is still under development, and Instok has recently started recruiting programmers to fulfil its product development needs. 

\subsection{The choice of client}
The CEO of Instok has approached me to consult them in developing their HR (Human Resource) Policies, as well as assist them in recruiting and interviewing candidates. This client has been chosen due to the scope of the project offered and the flexibility thereof. The CEO has shown the utmost understanding for the situation and is very willing to allow the researcher to be an integral part of his company to stimulate the progress of the research. On the other hand, other potential clients failed to meet that criterion or wanted a research that left no room for strategic thinking, which is of high importance to the study. For that reason, and looking at the timing of the study, Instok was chosen as client.
%I decided to consult this company because the opportunity to consult a company on the very fundamentals of HR Management seemed like a great personal challenge. The company is in a critical stage in its development and the opportunity to assist them in achieving their goals appealed to me greatly. This is also linked to me having an interest in seeing the results of my efforts. In a smaller organization this is generally easier. In the past few years I have discovered that I'm generally more attracted to smaller organizations, and this research may help many more than solely Instok.

\subsection{Problem Orientation}
In this subsection the problem of the client will be explored.\\

\subsubsection{Client problem}
As stated above, the client has recently begun its recruiting efforts in order to recruit personnel to both develop the product further as well as fulfil other functions. However, Instok, being a very young company, does not have a framework to recruit, hire, or maintain personnel apart from the founders of the company. \\
The CEO of the company indicated he's interested in starting small, thinking big. He said this means to make a system that can be applied to a company with only a few applicants and subsequently improved to suit the company's needs in the future.\\
Unfortunately, recent recruiting efforts have led to confusion inside the Instok team regarding the HR side of the company. Until now, the client hasn't had to need to recruit, hire, maintain, and, possibly, fire people. Due to their inexperience with the HR field Instok has the need for a consultant to consult them on the fundamentals of creating HR policies. At the moment, Instok is not organized in terms of HR and after this project it would like to have an idea on the fundamentals of HR policies and how to continue the formation of HR policies.

\subsubsection{General problem} It has been recognized that startups struggle with HR related issues more often than not (\parencite{FUNG2012},\parencite{WEISSMAN2016},\parencite{BUCH2016}). In the past there was a tendency to neglect HR in a startup, and this can lead to uncomfortable situations \parencite{WEISSMAN2016}. In many startups HR is seen as a bureaucratic activity that stifles the creativity and the dynamic environment of the company \parencite{FUNG2012}. But this is a stereotypical view of HR, and it's important for startups to realize that HR is much more than a necessary evil \parencite{WEISSMAN2016}. Startups that have ignored HR in the early stages of their development have had lower rates of success than their counterparts who developed their HR function in the early stages of the company's lifespan. However, contrary to popular belief, it's not the size of the startup that determines whether or not to take on an HR person. It's the growth of the company that determines whether or not to start thinking about HR \parencite{COY2015}. Recruitment is a tricky process that can lead a company to success if done right, and to its demise if done poorly \parencite{MUMBAI2014}. However, it's tempting to think HR exists solely for the purpose of hiring and firing \parencite{COY2015}. In fact, there is a lot more to consider when thinking about HR. \cite{COY2015} called it "Making things go right", it's the act of making sure everybody can perform to their best in the working environment offered. This widens the scope of HR considerably and thus it's important for a startup to think about HR in the widest sense of the function, and this function seems to be of great help to startups who are starting expanding their teams and want to get the best out of their employees.\\
But what influences HR policies? \cite{HRCA2016} stated that HR policies are primarily influenced by the law of the country that the company operates in (Brazil in this case). Secondly, it's important to listen to the employees of the company to develop HR policies that are appropriate for the company in question. In competitive markets, it's also recommended to look at peer companies to see what HR policies these companies use, as it can influence the recruitment market position of the company \parencite{HRCA2016}.\\
In conclusion, HR is an important function for startups that should not be neglected. In the case of Instok, it would be wise to explore the environment a bit further to have a better idea of the right time to form the right HR policies. It is also important for Instok to not only receive information about the environment in which they operate, but also to receive guidelines on how to apply this information in the form of HR policies. In the next section this study will be structured into a main research question and corresponding sub-questions. 

\section{Research Objective}
In order to consult Instok on its HR policies, it's important to create a primary objective for the project. In this project, HR policies are defined by \cite{HRCA2016} as: \\

\begin{displayquote}
"A policy is a formal statement of a principle or rule that members of an organization must follow. Each policy addresses an issue important to the organization's mission or operations."
\end{displayquote}

After conversing with the CEO of Instok I have come to the conclusion that the primary research question can be specified as:\\
"What are the HR policies Instok needs to develop to prepare the company for the future challenges in the HR field?"\\
%"The objective of this project is to consult Instok on which HR policies to develop to prepare the company for the future challenges in the HR field."\\
To answer this question, a set of questions needs to be answered. It must be understood that this project does not entail the creation of each and every HR Policy for Instok, it aims to to consult the company on the best way to create these policies based the items that will be listed later in this document. 

\subsection{Research questions}
As mentioned before, to reach the primary research objective a set of questions needs to be answered, namely:
\begin{enumerate}
\item Which HR policies does Brazilian law stipulate?
\item Which HR policies do the employees of Instok think important?
\item Which HR policies do peer companies use?
%\item What is the management style of the company?
%\item What is the target group of Instok regarding recruitment?
%\item What is the effect of the answers to the previous questions on the development of the HR function?
\end{enumerate}

These questions serve as the skeleton for the research. In the next section the Methodology (the way to answer these questions) will be explained.

%\subsection{Indicators of success}
%In dialogue with the CEO, a primary research objective has been established and after that the questions to reach the objective have been answered. Now how will we know if the project was a success? This subsection will explain which metrics will be used to measure the success of the project.\\


\section{Methodology}
The methodology of this research is aimed at creating a research that is feasible within the allotted time (until June 2017) while allowing for sufficient reliability and validity. An important item to start with is the way the client is organized. As stated earlier, the company is organized according to the principles laid out in the book by Eric Ries. This has a profound effect on how processes and projects are organized within the company \parencite{RIES2011}. This has an effect on how the processes should be developed, which, in turn, has an effect on the recommendation to develop the processes.\\
In order to create strategic HR policies that will serve the client in the long term, it's important to check what's most effective in other companies \parencite{SNL2013}. This research will look at which HR policies are the most relevant for Instok. It will do this by performing peer company research in order to get a more local picture, and by conducting literature research. \cite{HRCA2016} also stated that it's important to talk to the employees of the company to see if anything is not clear in the current situation of Instok that would benefit from an HR policy.

\subsection{Literature research}
In addition to the peer company research, literature research will be done in order to answer the remaining research questions: "Which HR policies does Brazilian law stipulate?" \\
%In addition to answering the above research question, the literature research will also search to confirm or contest the results of the semi-structured interviews. It aims to connect the results to the literature to generate an overview of the situation that is as complete as possible.	
The literature review will be a "Traditional or Narrative literature Review" as defined by the \cite{TOLEDO2016}. This means that the literature review will focus on the connection proposed by well-known theories between the way HR policies are developed and the characteristics of the companies they are developed in, with a focus on LEAN Startups \parencite{RIES2011}. The researcher will search the Google and Google scholar search engines for the following keywords: HR policies development, HR policies in startups, HR policies lean startup, HR policies in LEAN enterprises.

\subsection{Speaking to Instok employees}
Speaking to Instok employees aims to answer the sub-question: "Which HR policies do the employees of Instok think important?" Contact with the employees of Instok will provide data on the current situation within the client company and will establish attention areas for the HR policies. Because this topic has many facets, open interviews will be held. In these interviews employees will only be asked one question: "Is there anything in the Instok organization that is not clear to you?"\\
After this questions, the researcher will ask deepening questions on whatever subjects come up in order to get a better understanding of the issues.

\subsection{Peer company research}
Peer company research will be conducted to answer the following research question: "Which HR policies do peer companies use?" It will also aim to explore which policies Instok might need.\\

Curitiba is one of the cities in Brazil that boasts an impressive number of startups \parencite{SANTANA2016}.
Peer company research will be done amongst 2 groups of startups: one group that shares characteristics with Instok and another group that is focused on larger startups. \\

\subsubsection{Group 1} The first group of peer companies share the following characteristics with Instok (Group 1):
\begin{itemize}
\item The companies are <2 years old
\item The companies are active with an online product platform
\item The companies employ =< 15 employees
\item The companies operate according to the LEAN Startup framework as proposed by \cite{RIES2011}
\end{itemize}

\subsubsection{Group 2} The second group of peer companies have the following characteristics (Group 2):
\begin{itemize}
\item The companies are 2-4 years old
\item The companies are active with an online product platform
\item The companies employ > 15 employees
\item The companies operate according to the LEAN Startup framework as proposed by \cite{RIES2011}
\end{itemize}

One can imagine that this leaves a large number of startups in the sample. The optimal research method in this case would be interviews, since it's important to receive in-depth information about the situation of the companies and their HR policies. For in-depth research generally 20 interviews are necessary \parencite{BAKEREDWARDS2012}, this is supported by \cite{SNL2013}, who state that for meaningful analysis, a peer group should not be larger than 20 companies. In this case this means 10 companies for group 1 and 10 for group 2. To perform these interviews, the researcher will visit startup coworking spaces in Curitiba to find the companies that conform to the criteria mentioned above. The following startup hubs will be visited: Impact Hub Curitiba, IBQP Curitiba, Aldeia Coworking, Nex Coworking, and Open Office. These are all places within Curitiba that facilitate the growth and interaction of startups. \\

With a sample of these startups, semi-structured interviews will be held in Portuguese. Semi-structured interviews will allow for clear questions to be asked by the research, while facilitating the interviewee to mention items that were previously not included in the interview. In this situation, where every company is different and various themes are present, a structured interview might miss valuable data. The following questions will be asked to the peer companies, these questions are based on the research objective and the sub-questions as stated earlier in this document.
\begin{itemize}
\item Does your company have HR policies? Why/why not?
\item What kind of HR policies are present?
\item If no HR policies are present, how are agreements regarding HR items made inside your company?
\item What was the creation of the HR policies in your company like?
%\item What is the management style of your company?
%\item What is the target group of your company regarding recruitment?
\end{itemize}

In addition to these questions, the researcher also reserves the right to ask questions to clarify items that come up in the interview. The researcher will take note of the items that come up in the interview. In qualitative research people often suggest recording and transcribing the interviews (\parencite{HUMBLE}, \parencite{BAILEY2008}, \parencite{THOMSON2014}). However, since in this research it is less important to look at the specific behaviour of the people and the questions are more focused on factual data, notes will suffice.

\section{Chronograph}
The project is planned to be concluded on June 30, 2017 with Instok. The following table provides a chronograph of the project.

\begin{table}[]
\centering
\caption{Chronograph of the project}
\label{chronograph}
\begin{tabular}{|l|l|l|}
\hline
Item                             & Start Date & End Date   \\ \hline
Preparing the proposal			 & 01/02/2017 & 13/03/2017 \\ \hline
Interviews with Instok Employees & 15/03/2017 & 22/03/2017 \\ \hline
Literature Research              & 23/03/2017 & 10/05/2017 \\ \hline
Interviews with  peer companies  & 10/05/2017 & 10/06/2017 \\ \hline
Writing final report             & 10/06/2017 & 30/06/2017 \\ \hline
\end{tabular}
\end{table}

\printbibliography

%\appendix


\end{document}

%% 
%% Copyright (C) 2011-2016 by Brian D. Beitzel <brian at beitzel.com>
%% 
%% This work may be distributed and/or modified under the
%% conditions of the LaTeX Project Public License (LPPL), either
%% version 1.3c of this license or (at your option) any later
%% version.  The latest version of this license is in the file:
%% 
%% http://www.latex-project.org/lppl.txt
%% 
%% Users may freely modify these files without permission, as long as the
%% copyright line and this statement are maintained intact.
%% 
%% This work is not endorsed by, affiliated with, or probably even known
%% by, the American Psychological Association.
%% 
%% This work is "maintained" (as per LPPL maintenance status) by
%% Brian D. Beitzel.
%% 
%% This work consists of the file  apa6.dtx
%% and the derived files           apa6.ins,
%%                                 apa6.cls,
%%                                 apa6.pdf,
%%                                 README,
%%                                 APAamerican.txt,
%%                                 APAbritish.txt,
%%                                 APAdutch.txt,
%%                                 APAenglish.txt,
%%                                 APAgerman.txt,
%%                                 APAngerman.txt,
%%                                 APAgreek.txt,
%%                                 APAczech.txt,
%%                                 APAturkish.txt,
%%                                 APAendfloat.cfg,
%%                                 apa6.ptex,
%%                                 TeX2WordForapa6.bas,
%%                                 Figure1.pdf,
%%                                 shortsample.tex,
%%                                 longsample.tex, and
%%                                 bibliography.bib.
%% 
%%
%% End of file `./samples/longsample.tex'.
